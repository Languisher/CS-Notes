
\chapter{Graphes eulériens et hamitoniens} % (fold)
\label{chap:Graphes eulériens et hamitoniens}

% chapter Graphes eulériens et hamitoniens (end)

\section{Chemin eulériens et graphe eulérien} % (fold)
\label{sec:Chemin eulériens et graphe eulérien}

% section Chemin eulériens et graphe eulérien (end)

\subsection{Rappels} % (fold)

% subsection Rappel : Chemin élémentaire et chemin simple (end)
Rappel : Définition d'un \textbf{chemin élémentaire} et \textbf{chemin simple} : 
\begin{itemize}

  \item \textbf{Élémentaire} : s'il passe au plus une fois par chaque sommet. 

  \item \textbf{Simple} : s'il passe au plus une fois par chaque arête.

\end{itemize}


On a vu : Longueur minimale $\implies$ élémentaire $\implies$ simple. (Mais les réciproques sont fausses)

Rappel : Un \textbf{cycle} est un chemin simple reliant un sommet à lui même de longueur non nulle (donc $n \ge 3$)

\subsection{Nombre de chemin reliant 2 sommets} % (fold)
\label{sub:Nombre de chemin reliant 2 sommets}

% subsection Nombre de chemin reliant 2 sommets (end)
\textbf{Question} : Combien y-a-t'il de chemins reliant 2 sommets ? Si les chemins ne sont pas simples, il y en a une infinité. Donc on cherche le nombre de chemins simples. 



Rappel : La \textbf{matrice d'adjacence} d'un graphe $G = (S, A)$ dont on numéroté les sommets $S = \{S_1, S_2, \dots, S_n \}$ où $n$ l'ordre de $G$ est définie par : 
\begin{equation}
  M = \left( m _{i,1} = \begin{cases}
      1 \text{ si } \{ s_i, s_j \} \in A \\ 
      0 \text{ sinon}
  \end{cases} \right)  \in \mathrm{M} _n (\{0, 1\})
\end{equation}

On a : 
\begin{equation}
  M ^{2} = M \times M = \left( \sum_{k=1}^{n} m _{i,k} m _{k,j}\right) _{1 \le i, s \le n} \in   \mathrm{M} _ n (\mathbb{K})
\end{equation}

Or : 
\begin{equation}
  m _{i,k} m _{k,j} = \begin{cases}
    1 \text{ si } \{ s_i, s_k \} \in A \text{ et } \{s_k, s_j\} \in A \\
    0 \text{ sinon}
  \end{cases}
\end{equation}

Donc $m _{i,k} m _{k,j} = 1$ si et seulement si $s_i$ et $s_j$ sont reliés par le chemin $(s_i, s_k, s_j)$ de deux arêtes. 

Donc 
\begin{equation}
  \sum_{k=1}^{n} m _{i,k} m _{k,j} = \text{ nombre de chemins de deux arêtes reliant } s_i \text{ et } s_j
\end{equation}

Plus généralement, pour tout $p \in \mathbb{N} ^{*}$, si on note 
\begin{equation}
  M ^{p}= \left( m _{i,j} ^{p} \right) _{1 \le i, j \le p} \in \mathrm{M} _n (\mathbb{R})
\end{equation} 


alors $m_{i,j} ^{p}$ = \underline{nombre de chemins de $p$ arêtes reliant $s_i$ et $s_j$}

\textbf{Problème} : Cette méthode premet de dénombrer \underline{tous} les chemins de $p$ arêtes reliant $s_i$ à $s_j$, mais pas seulement les chemins simples. 

\begin{Theorem}{}{}
Le nombre de cycles de longueur 3 (on les appelle les \underline{triangles}) est égal 
\begin{equation}
\frac{1}{3!}   \mathrm{Tr}(\mathrm{M} ^{3}) = \sum_{k=1}^{n} = \frac{1}{6}  \sum_{k=1}^{n} m _{k,k} ^{3}
\end{equation}
\end{Theorem}

\begin{myproof}{}{}
On a vu que $m _{k,k} ^{3}$ = nombre de chemins de 3 arêtes reliant $s_k$ à lui-même. 

Ces chemins sont nécessairement simples, donc ce sont des cycles de longueur 3.

Il y en a au total 
\begin{equation}
  \sum_{k=1}^{n} m _{k, k} ^{3} = \mathrm{Tr} ( \mathrm{M} ^{3})
\end{equation}

Or chaque cycle de longueur 3 passe par 3 sommets.

Il est donc compté $3! = 6$ fois dans la somme. (Nombre de permutations des 3 sommets)

Au final, on obtient $\mathrm{Tr}(\mathrm{M} ^{3}) / 6$ triangles.
\end{myproof}

\begin{Definition}[colbacktitle=red!75!black]{Eulèrien}{}
Un chemin est dit \textbf{eulèrien} s'il passe \underline{exactement une seule fois} par chque sommet du graphe. 

Un graphe est dit \textbf{eulèrien} s'il contient un cycle eulèrien.
\end{Definition}

\textbf{Remarque} : Un chemin \underline{eulèrien} est nécessairement \underline{simple}.

\textbf{Remarque}: Un chemin \underline{eulèrien} n'est pas nécessairement un \underline{cycle}.

On peut montrer que ce graphe ne contient pas de \underline{cycle eulèrien} donc ce graphe n'est pas eulèrien.


\begin{Theorem}{}{}
Un graphe est eulèrien si et seulement s'il est connexe et tous ses sommets sont de degré pair.
\end{Theorem}

\begin{Example}{}{}
Dans notre exemple, $d(c) = 2$ et $d(c) = d(d) = 4$ sont pairs mais $d(a) = d(b) = 3$ sont impairs donc le graphe n'est pas eulèrien.
\end{Example}

\begin{myproof}{}{}
$\implies$ Si un graphe est eulèrien alors c'est un cycle eulèrien ()
\end{myproof}













