\chapter{Operators and Expressions} % (fold)
\label{chap:Operators and Expressions}

\section{L-values and R-values} % (fold)
\label{sec:L-values and R-values}
\begin{Definition}[colbacktitle=red!75!black]{L-values and R-values}{}
An \textbf{L-value} is something that can appear on the left side of an equal sign. 
An \textbf{R-value} is something that can appear on the right side of an equal sign. 
\end{Definition}

\begin{Prop}{L-values and R-values}{}
\begin{enumerate}

    \item An \textbf{L-value} identifies a \underline{specific location},
      \begin{itemize}

          \item where a result can be stored 
          \item that we can refer to later in the location

      \end{itemize}
    \item An \textbf{R-value} designates a \underline{value}.
\end{enumerate}
\end{Prop}

\begin{Example}{}{}
\begin{lstlisting}
b + 25 = a; // Wrong, we can't predict where the result will be. 

int a[30]; 
a[ b + 10 ] = 0; // True, we do know where *(a + b + 10) would be !

int a, *pi; 
pi = &a; // True, the expression specifies the location to be modified.
\end{lstlisting}

The value in the pointer \verb|pi| is the address of a specific location in memory, and the \verb|*| operator directs the machine to that location. 
\begin{itemize}

    \item When used as an \textbf{L-value}, this expression specifies the location to be modified.

    \item When used as an \textbf{R-value}, it gets the value currently stored as that location.
\end{itemize}
\end{Example}

\begin{Prop}{Meaning of exp1=exp2}{}
Here, we take the \textbf{address} of \verb|exp1| \textbf{(L-value)} and the \textbf{value} of \verb|exp2| \textbf{(R-value)}.

We pass the \textbf{value} \verb|exp2| into the \textbf{value} which \textbf{position} \verb|exp1| possess.
\end{Prop}


\section{Operator precedence} % (fold)
\label{sec:Operator precedence}

% section Operator precedence (end)

% section L-values and R-values (end)
% chapter Operators and Expressions (end)
